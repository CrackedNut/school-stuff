\documentclass[a4paper,12pt]{article}

\usepackage{amsmath}
\usepackage{amsthm}

\begin{document}
    \newtheorem*{osservazione}{Osservazione}
    \newtheorem*{definizione}{Definizione}
    \newtheorem*{notazione}{Notazione}
    \newtheorem{proposizione}{Proposizione}[section]
    \newtheorem{corollario}{Corollario}[proposizione]
    \title{Formalizzazione della regola della mensa}
    \author{Luigi Pugliese}
    \maketitle
    \begin{abstract}
        La regola della mensa è una delle fondamentali leggi che governano la vita degli studenti dell' Università della Calabria che per sfamarsi utilizzano i servizi offerti dalle varie mense.
        Essa dà il diritto di sottrarre \emph{una} patatina per piatto da ognuno dei commensali di chi la applica. Può essere estesa al di fuori del proprio tavolo, ma questo articolo tratterà il caso specifico
        in cui la regola è applicabile solo al proprio tavolo.

        Questo articolo ne darà una formulazione matematica.
    \end{abstract}
    \newpage
    \renewcommand\contentsname{Contenuti}
    \renewcommand\proofname{Dimostrazione}
    \tableofcontents{}
    \addtocontents{toc}{~\hfill\textbf{Pagina}\par}
    \newpage
    \section{Assiomi}
    Alcuni assiomi della mensa sono indispensabili.
    \begin{enumerate}
        \item No tesserino, no mangi (se non dividi)
        \item Ogni studente può passare quanti tesserini vuole
        \item Ogni tesserino ha 5 punti a disposizione
        \item Per ogni tesserino esiste un solo primo e un solo secondo
        \item Chi ci lavora è \emph{quasi} sempre stronzo
    \end{enumerate}
    \newpage
    \section{Commensali e patatine}
    \subsection{Cosa è un commensale}
    \begin{definizione}
        Sia $I$ l'individuo che vuole applicare la regola della mensa. Per commensali si intendono le persone sedute allo stesso tavolo di $I$ e possiamo indicare con $C$ l'insieme dei commensali.
    \end{definizione}
    \begin{osservazione}
        Il numero dei commensali è finito, dunque possiamo numerarli, ossia assegnare a ognuno di esse un numero naturale, dunque indichiamo il signolo commensale con $c_j$.
    \end{osservazione}
    \begin{osservazione}
        Per l'assioma $\mathbf{2}$, possiamo supporre che un commensale abbia $t$ tesserini a disposizione    
    \end{osservazione}
    \subsection{I piatti di patatine}
    \begin{definizione}
        Un piatto di patatine è un qualsiasi piatto contenente almeno una patatina. Indichiamo l'insieme dei piatti di patatine su un tavolo con $\mathbf{P}$.
    \end{definizione}
    \begin{proposizione}
       Sia $t$ il numero di tesserini posseduti da una persona. Allora quella persona può avere al più $2t$ piatti di patatine.
    \end{proposizione}
    \begin{proof}
        Ovviamente se $t=0$, allora la persona non può possedere propriamente alcun piatto, dunque non può avere piatti contententi patatine.
        Supponiamo ora $t>0$. Per gli assiomi $\mathbf{4}$ e $\mathbf{5}$, non puoi avere due secondi (per tesserino), dunque, poiché l'unica altra alternativa per avere più piatti di patatine sono i contorni, si ha al più $2$ piatti di patatine, in quanto due contorni con patatine non te li danno.
    \end{proof}
    \begin{corollario}
        Sia $|C|=n$. Ogni tavolo può avere al più $2nt$ piatti di patatine.
    \end{corollario}
    \begin{proof}
        Per la proposizione precedente, ogni commensale può avere al più $2t$ piatti di patatine, dunque se $|C|=n$, si avranno su uno stesso tavolo al più $2tn$ piatti di patatine. 
    \end{proof}
    \newpage
    \subsection{Le patatine}
    \begin{osservazione}
        Ogni elemento di $\mathbf{P}$ ha un numero \underline{molto} finito di patatine, dunque possiamo numerarle.
    \end{osservazione}
    \begin{notazione}
        Possiamo indicare ogni singola patatina con $p_{i,j}$, dove $i$ indica il piatto in cui si trova e $j$ il numero naturale assegnatole.\\
        Inoltre, possiamo costruire una matrice $\mathbf{M}$ di dimensione $a\times b$, dove $a=|P|$ e $b=max\{|\pi|:\pi\in\mathbf{P}\}$. Si avrà:
        \begin{center}
            $\mathbf{M}=
            \begin{bmatrix}
                & p_{1,1} & \dots & p_{1,b}\\
                &\vdots\\
                & p_{a,1} & \dots & p_{a,b}\ (patatina\ o\ \emptyset )
            \end{bmatrix}$
        \end{center}
        costruita in modo che, se esiste $\pi\in\mathbf{P}$ tale che $|\pi|<max\{|\mu|:\mu\in\mathbf{P}\}$, ci saranno $b-|\pi|$ insiemi vuoti nella riga corrispondente al piatto $\pi$.
    \end{notazione}
    \section{Patatine ai commensali}
    \subsection{L'idea}
    \begin{definizione}
        Sia A l'insieme di tutte le patatine unito a $\{\emptyset\}$, l'insieme contenente \underline{come elemento} l'insieme vuoto.
    \end{definizione}    
    L'idea è quella di assegnare a ogni commensale una e una sola patatina da ogni piatto presente su un dato tavolo.
    \subsection{La mappa}
    A tale scopo possiamo utilizzare la seguente mappa:
    \begin{center}
        $\begin{matrix}
        \Pi&\!\!\!\!:\mathbf{C}\longrightarrow\ A\times A\times\dots \times A\ (b\ volte)\\
           &\!\!\!\!\!\!\!\!\!\!\!\!\!\!\!\!\!\!\!\!\!\!\!\!\!\!\!\!\!\!\!\!\!\!\!\!c_i \longmapsto (p_{1,j},..., p_{a,j})
        \end{matrix}$
    \end{center}
    dove $\mathbf{C}$ è l'insieme dei commensali e i $p_{n,j}$ sono gli elementi della matrice $\mathbf{M}$.
    \subsection{Perché funziona?}
    Questa funzione è una buona formulazione della regola della mensa? Sì, infatti l'arbitrarietà della numerazione di ogni elemento ci assicura che ogni patatina di ogni piatto venga assegnata a uno dei commensali e in caso non ci siano sufficienti patatine, verrà assegnato l'insieme vuoto, ossia nessuna patatina.
    \section{Conclusioni}
    Questa formulazione della regola della mensa, seppur limitata, è facilmente ampliabile e permette una pacifica convivenza fra gli utilizzatori della mensa. 
    \newpage
    \section{Ringraziamenti}
    Ringrazio Ismallo per l'idea, Narco che mi è stato accanto tutto il pomeriggio, internet che mi ha insegnato a scrivere in $LaTe\chi$ e il mio PC che non si è scaricato.
\end{document}